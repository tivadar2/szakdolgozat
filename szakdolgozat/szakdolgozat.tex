\documentclass[12pt]{article}
\usepackage[a4paper]{geometry}
\usepackage[magyar]{babel}
\usepackage[utf8]{inputenc}
\usepackage{graphicx}
\usepackage{t1enc}
\usepackage{amsmath}
\usepackage{epstopdf}
\usepackage{hyperref}
\usepackage{indentfirst}
\usepackage{float}
\usepackage{caption}
\usepackage{cleveref}
\usepackage{subcaption}
\usepackage{multirow}
\usepackage{setspace}
\title{}
\date{}
\begin{document}

% TODO: komplexitásszerű dolgok leírása, hányszor kell fitness-t számolni, a különböző módszereknél

\begin{titlepage}
	\centering
	{\Huge\bfseries Szakdolgozat\par}
	\vspace{1cm}
	\vspace{1cm}
	{\LARGE Életkor becslése szociális hálózatok gépi optimalizációs módszerekkel\par} % TODO: ez marad a cím?
	\vspace{3cm}
	{\huge\bfseries Pongó Tivadar\\\par}
	\vspace{3cm}
	\begin{table}[H]
		\centering
		\begin{tabular}{ccl}
			&	\LARGE\textbf{Témavezető:} &\Large Dr. Török János \\ % TODO: ezek jók-e?
			& &\\
			& &\Large egyetemi docens \\
			& &\\
			& &\Large BME Elméleti Fizika Tanszék \\
			
		\end{tabular}
		\caption*{}
		\label{t1}
	\end{table}
	\vfill
	\begin{figure}[H]
		\centering
		\includegraphics[scale=0.35]{img/bme_logo_kicsi.eps}
	\end{figure} 
	{\large Budapesti Műszaki és Gazdaságtudományi Egyetem\\ 2017 \par}
\end{titlepage}
\onehalfspacing
\iffalse
\section*{Szakdolgozat}
\begin{itemize}
	\item a TDK alapján megírtam a programot, ebből mennyit kell majd leírnom szakdolgozatba?
	\item időzabáló rész a gauss görbe a konvolúciónál (maga a konvolúció elég sok idő) 20-30 s egy futás a kb. 11000 ego-ra + gauss optimalizálása listával
	\item Gradiens módszer egy és több dimenzióban
	\begin{itemize}
		\item "zajos" a pm2 függvény, emiatt derivált számolásnál nagy "dx"-et kell választani -> ábra majd
		\item gamma helyes megválasztása, konstans vagy függ a gradienstől (a konstans működött)
		\item először az $\dfrac{1}{s + \sigma}$ súlyt próbáltam, ahol s a paraméter, $\sigma$ pedig a csoport életkorainak szórása
		\item 3 felé osztottam a csoport életkorainak szórása alapján a súlyfaktort 3 alatt, 3 és 6 közt és 6 felett, így futtatva a gradiens módszert a 6 feletti szórások súlyára negatívat kaptam -> lehet, hogy így több csúcs lesz a negatív súly miatt?
	\end{itemize}
	\item Simulated annealing
	\item genetikus algoritmus
\end{itemize}
pm2 függvény - plusz mínusz 2 éven belüli becslések aránya, 0 és 1 közti szám \\ \\
Margók mérete a latex alapbeállítása?
\fi
\pagebreak
\tableofcontents
\pagebreak
\section{Bevezetés}
\subsection{Motiváció}
Napjainkban a felgyorsult információáramlás miatt rengeteg adat keletkezik rólunk, illetve környezetünkről az interneten. % TODO: vagy máshol is
Hatalmas mennyiségről van szó, mely akár publikus is lehet, bár legtöbbször valamilyen jogi személy (például vállalatok, állam, szervezetek) birtokában áll. Az utóbbi években nagyon felkapott lett ezek feldolgozása, azaz, hogy ebből a sok adatból hasznos információkat lehessen leszűrni.
% TODO: big data
% TODO: milyen nagyszerű dolgokat csináltak ezzel
% TODO: én mit akarok ezzel csinálni (optimalizáció)
% TODO: komplex rendszer, tehát fizikával összehasonlítható

% TODO: E/1-ben írjak, vagy T/1-ben

\subsection{Az életkorbecslő eljárás} % TODO: hogyan hivatkozzak a TDK-ra?
Az életkor meghatározásához alapvetően egy szociális hálózatot használunk fel. Ezen belül úgynevezett egocentrikus hálózatot, amely már csak egy adott személy kapcsolatrendszerét tartalmazza. Az adatbázisban a kapcsolatok mellett az ismerősök életkorait tartalmazzák. Persze ezek az adatok hiányosak, vagy lehetnek hibásak is. A kapcsolati hálózat egy gráfként kezelhető, amelyben a csúcsok az embereknek, az élek pedig a köztük lévő ismeretségnek felelnek meg.

Az embereket jól jellemzik a közösségek, amelyeknek tagjai. Ilyenek például az osztályok, a családok, a munkatársi vagy baráti körök. Egy kiválasztott egyén életkora összefüggésbe hozható a közösségei átlagéletkoraival. Ez az összefüggés nem triviális, viszont felhasználható a vizsgált személy korának meghatározásához. Ehhez először fel kell térképezni a közösségeket a hálózatban, ami nem egyértelmű feladat. Létezik \iffalse TODO: létezik helyett valami más szó \fi többfajta közösségfelismerő algoritmus, melyek megpróbálják az ember által elképzelt közösségképet felhasználni a kereséshez, illetve a hálózat esetleges hibáit is figyelembe venni (például hiányzó élek vagy csúcsok). Az általunk továbbiakban használt csoportok az úgynevezett klikk perkolációs algoritmussal \cite{tamas_gabor_tdk} lettek feltérképezve. % TODO: vagy másik cikk
Ezután készítünk egy hisztogramot, még pedig úgy, hogy a csoportok átlagéletkoraira centrált Gauss-görbéket összegezzük az összes csoportra, tehát egy diszkrét konvolúciót végzünk el.
\begin{equation} \label{simito_gauss}
	g(x) = \frac{1}{\sqrt{2\pi}\sigma}e^{-\frac{x^2}{2\sigma^2}}
\end{equation}
\begin{equation} \label{diszkret_konv}
	K(a) = \sum_{i=1}^{C} g(a-a_i) \cdot p(n_{i}, \sigma_{i}) % TODO: kell-e részletezni, hogy ez miért konv.?
\end{equation}
Az \iffalse TODO: Az-A \fi (\ref{simito_gauss}) képlet egy normált Gauss-görbe, amelynek a szórásparamétere $\sigma$. A \iffalse Az-A \fi (\ref{diszkret_konv}) képletben $a$ az életkort jelöli, $C$ a csoportok számát, $a_i$ az i. csoport átlagéletkorát, $p$ pedig egy súlyfaktort, amely függhet az adott csoport nagyságától ($n_{i}$) és a csoporton belüli életkorok szórásától ($\sigma_i$). Alapesetben $p=1$, ekkor nem vesszük figyelembe a csoportok minőségét\iffalse TODO: milyenségét \fi. Alább \aref{hisztogram_pelda}. ábrán láthatunk egy példát a $K(a)$ függvényre.
\begin{figure}[H]
	\centering
	\includegraphics[scale=1.0]{}
	\caption{Hisztogram} % TODO: bővebb aláírás
	\label{hisztogram_pelda}
\end{figure}
\subsection{Optimalizáció jelentése}
\section{Optimalizációs módszerek}
\subsection{Gradiens módszer}
\subsection{Szimulált hűtés}
\subsection{Genetikus algoritmus}
\section{Eredmények}
\begin{thebibliography}{9}
	\bibitem{tamas_gabor_tdk}
	TDK % TODO: rendesen hivatkozni
\end{thebibliography}
\end{document}