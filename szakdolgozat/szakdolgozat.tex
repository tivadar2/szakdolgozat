\documentclass[12pt]{article}
\usepackage[a4paper]{geometry}
\usepackage[magyar]{babel}
\usepackage[utf8]{inputenc}
\usepackage{graphicx}
\usepackage{t1enc}
\usepackage{amsmath}
\usepackage{epstopdf}
\usepackage{hyperref}
\usepackage{indentfirst}
\usepackage{float}
\usepackage{caption}
\usepackage{cleveref}
\usepackage{subcaption}
\usepackage{multirow}
\usepackage{setspace}
\usepackage{diagbox}
\usepackage{setspace}
\usepackage{natbib}
\bibliographystyle{unsrt}
%\usepackage[backend=biber]{biblatex}
%\addbibresource{bibliography.bib}


\begin{document}

% TODO: komplexitásszerű dolgok leírása, hányszor kell fitness-t számolni, a különböző módszereknél
% TODO: margók
% TODO: hogyan hivatkozzam az egész életkorbecslő dologra?
% TODO: negatív csúcsok -> jobb v-jű lesz a többi csúcs?

\begin{titlepage}
	\centering
	{\Huge\bfseries Szakdolgozat\par}
	\vspace{1cm}
	\vspace{1cm}
	{\LARGE Életkor becslése szociális hálózatok gépi optimalizációs módszerekkel\par} % TODO: ez marad a cím?
	\vspace{3cm}
	{\huge\bfseries Pongó Tivadar\\\par}
	\vspace{3cm}
	\begin{table}[H]
		\centering
		\begin{tabular}{ccl}
			&	\LARGE\textbf{Témavezető:} &\Large Dr. Török János \\
			& &\\
			& &\Large egyetemi docens \\
			& &\\
			& &\Large BME Elméleti Fizika Tanszék \\
			
		\end{tabular}
		\caption*{}
		\label{t1}
	\end{table}
	\vfill
	\begin{figure}[H]
		\centering
		\includegraphics[scale=0.35]{img/bme_logo_kicsi.eps}
	\end{figure} 
	{\large Budapesti Műszaki és Gazdaságtudományi Egyetem\\ 2017 \par}
\end{titlepage}
\onehalfspacing
\iffalse
\section*{Szakdolgozat}
\begin{itemize}
	\item a TDK alapján megírtam a programot, ebből mennyit kell majd leírnom szakdolgozatba?
	\item időzabáló rész a gauss görbe a konvolúciónál (maga a konvolúció elég sok idő) 20-30 s egy futás a kb. 11000 ego-ra + gauss optimalizálása listával
	\item Gradiens módszer egy és több dimenzióban
	\begin{itemize}
		\item "zajos" a pm2 függvény, emiatt derivált számolásnál nagy "dx"-et kell választani -> ábra majd
		\item gamma helyes megválasztása, konstans vagy függ a gradienstől (a konstans működött)
		\item először az $\dfrac{1}{s + \sigma}$ súlyt próbáltam, ahol s a paraméter, $\sigma$ pedig a csoport életkorainak szórása
		\item 3 felé osztottam a csoport életkorainak szórása alapján a súlyfaktort 3 alatt, 3 és 6 közt és 6 felett, így futtatva a gradiens módszert a 6 feletti szórások súlyára negatívat kaptam -> lehet, hogy így több csúcs lesz a negatív súly miatt?
	\end{itemize}
	\item Simulated annealing
	\item genetikus algoritmus
\end{itemize}
pm2 függvény - plusz mínusz 2 éven belüli becslések aránya, 0 és 1 közti szám \\ \\
Margók mérete a latex alapbeállítása?
\fi


\pagebreak
% Önállósági nyilatkozat
\begin{center}
	{\Huge Önállósági nyilatkozat}
\end{center}
\medskip
Alulírott Pongó Tivadar a Budapesti Műszaki és Gazdaságtudományi Egyetem fizika BSc szakos hallgatója kijelentem, hogy ezt a szakdolgozatot meg nem engedett segédeszközök nélkül, önállóan, a témavezető irányításával készítettem, és csak a megadott forrásokat használtam fel. \\
Minden olyan részt, melyet szó szerint, vagy azonos értelemben, de átfogalmazva más forrásból vettem, a forrás megadásával jelöltem.\\\\
Budapest, 2017. 05. 25. \\ \\ \\ %TODO: Dátum pontosan
\hspace*{100mm}aláírás
\pagenumbering{gobble}
\pagebreak
\tableofcontents
\pagebreak
\section{Bevezetés}
\pagenumbering{arabic}
Napjainkban a felgyorsult információáramlás miatt rengeteg adat keletkezik rólunk, illetve környezetünkről az interneten. % TODO: vagy máshol is
Hatalmas mennyiségről van szó, mely akár publikus is lehet, bár legtöbbször valamilyen jogi személy (például vállalatok, állam, szervezetek) birtokában áll. Az utóbbi években nagyon felkapott lett ezek feldolgozása, azaz, hogy ebből a sok adatból hasznos információkat lehessen leszűrni.
% TODO: big data
% TODO: milyen nagyszerű dolgokat csináltak ezzel
% TODO: én mit akarok ezzel csinálni (optimalizáció)
% TODO: komplex rendszer, tehát fizikával összehasonlítható
% TODO: erre akarok egy programot írni

% TODO: E/1-ben írjak, vagy T/1-ben

\section{Az életkorbecslő eljárás} % TODO: hogyan hivatkozzak a TDK-ra? + más cím itt?
Az életkor meghatározásához alapvetően egy szociális hálózatot használunk fel. Ezen belül úgynevezett egocentrikus hálózatot, amely már csak egy adott személy kapcsolatrendszerét tartalmazza. Az adatbázisban a kapcsolatok mellett az ismerősök életkorait tartalmazzák. Persze ezek az adatok hiányosak, vagy lehetnek hibásak is. A kapcsolati hálózat egy gráfként kezelhető, amelyben a csúcsok az embereknek, az élek pedig a köztük lévő ismeretségnek felelnek meg.

% TODO: adatbázisok
\subsection{Optimizálatlan életkor-meghatározás} \label{eletkor_becsles_tamas_gabor}
Ebben az alrészben (\ref{eletkor_becsles_tamas_gabor}) a \cite{tamas_gabor_tdk} TDK dolgozatban leírt életkorbecslő eljárást ismertetem, amely alapul szolgál ezen szakdolgozatnak.

Az embereket jól jellemzik a közösségek, amelyeknek tagjai. Ilyenek például az osztályok, a családok, a munkatársi vagy baráti körök. Egy kiválasztott egyén életkora összefüggésbe hozható a közösségei átlagéletkoraival. Ez az összefüggés nem triviális, viszont felhasználható a vizsgált személy korának meghatározásához. Ehhez először fel kell térképezni a közösségeket a hálózatban, ami nem egyértelmű feladat. Létezik \iffalse TODO: létezik helyett valami más szó \fi többfajta közösségfelismerő algoritmus, melyek megpróbálják az ember által elképzelt közösségképet felhasználni a kereséshez, illetve a hálózat esetleges hibáit is figyelembe venni (például hiányzó élek vagy csúcsok). Az általunk továbbiakban használt csoportok az úgynevezett klikk perkolációs algoritmussal \cite{tamas_gabor_tdk} \iffalse TODO: kell ez? \fi lettek feltérképezve. % TODO: vagy másik cikk
Ezután készítünk egy hisztogramot, még pedig úgy, hogy a csoportok átlagéletkoraira centrált Gauss-görbéket összegezzük az összes csoportra, tehát egy diszkrét konvolúciót végzünk el \eqref{diszkret_konv}.
\begin{equation} \label{simito_gauss}
	g(x) = \frac{1}{\sqrt{2\pi}\sigma}e^{-\frac{x^2}{2\sigma^2}}
\end{equation}
\begin{equation} \label{diszkret_konv}
	K(a) = \frac{1}{n_{sm}(a)}\sum_{i=1}^{C} g(a-a_i) \cdot s(n_{i}, \sigma_{i}) % illetve a simított életkoreloszlással való leosztás
\end{equation} % TODO: miért kell a simítás, egy-egy egyénre lehet, hogy nem az a jó, mint az összesre, 
\Aref{simito_gauss}. képlet egy normált Gauss-függvény, amelynek a szórásparamétere $\sigma$. \Aref{diszkret_konv}. képletben $a$ az életkort jelöli, $C$ a csoportok számát, $a_i$ az i. csoport átlagéletkorát, $s$ pedig egy súlyfaktort, amely függhet az adott csoport nagyságától ($n_{i}$) és a csoporton belüli életkorok szórásától ($\sigma_i$). Alapesetben $s=1$, ekkor nem vesszük figyelembe a csoportok milyenségét\iffalse TODO: minőségét \fi. A szumma előtti $n_{sm}$ egy normált, simított életkoreloszlás függvény, amellyel azért osztunk le, hogy kompenzáljuk a középkorúak felülreprezentáltságát a hálózatban. 
A konvolúció (alapesetben) $\sigma = 2$-es Gauss-al simítja el a csoport-átlagéletkor eloszlást, ami azért jó, mert így a közeli átlagéletkorú csoportok száma átlagolódik, súlyozva az évkülönbséggel. \iffalse TODO: ez lehet nem túl világos \fi Alább \aref{hisztogram_pelda}. ábrán láthatunk egy példát a $K(a)$ függvényre. A függvényt csak egész évű pontokban értékeltük ki.
\begin{figure}[H]
	\centering
	\includegraphics[scale=0.8]{img/82152624_age47.eps}
	\caption{Példa hisztogram} % TODO: bővebb aláírás
	\label{hisztogram_pelda}
\end{figure}
A hisztogramokon csúcsok találhatóak, amelyek azt jelentik, hogy van olyan csoport, amelynek átlagéletkora a csúcs helyén van. Minél nagyobb a csúcs, annál több ilyen csoport van. Ez alapján a vizsgált személy életkorát a legmagasabb csúcs helyével becsülhetnénk. Azonban nem mindegyik hisztogram olyan egyszerű, mint \aref{hisztogram_pelda}. ábrán látható, sokszor több csúcs hasonló nagyságú, ekkor érdemes figyelembe venni a csúcsok félértékszélességét. Definiálhatunk egy csúcs jósági tényezőt \aref{csucs_josag}. képlet szerint:
\begin{equation} \label{csucs_josag}
	v = \frac{h}{\textnormal{FWHM}},
\end{equation}
ahol $h$ a csúcs magassága, $\textnormal{FWHM}$ a csúcs félértékszélessége. Tehát minél magasabb és minél keskenyebb egy csúcs, annál jobb. Ezek után a program sorba rendezi a csúcsokat $v$ szerint és a legnagyobb $v$ értékkel rendelkezőt választja becsült életkornak.

\subsection{Paraméterek a becslés során}
Az életkor meghatározása során használunk különböző paramétereket, ilyenek például a Gauss-függvényben \eqref{simito_gauss} a $\sigma$\iffalse TODO: na jó nem biztos, hogy így szép \fi, illetve \aref{diszkret_konv}. képletben lévő $s(n_i, \sigma_i)$-ből származtatott paraméterek. Alapesetben $\sigma = 2$, $s = 1$, ezt optimalizálatlan esetnek nevezhetjük.

Többféleképpen is lehet definiálni paramétereket. Feloszthatjuk az $(n_i, \sigma_i)$ teret \iffalse TODO: vagy mit \fi tartományokra és az egyes tartományokon alkalmazott súlyok lesznek a paraméterek. A felosztást \aref{felosztas}. táblázat szerint végeztem, illetve külön vettem a $\sigma_{cs} = 0$ esetet. %TODO: befejezni%
\begin{table}[H]
	\centering
	\setstretch{1.5}
	\begin{tabular}{|c|c|c|c|}
		\hline
		\diagbox{$\sigma_{cs}$}{$n_{cs}$} & $]0;5]$ & $]5;10]$ & $]10;\infty[$  \\
		\hline
		$]0;3]$ & $p_1$ & $p_2$ & $p_3$ \\
		\hline
		$]3;6]$ & $p_4$ & $p_5$ & $p_6$ \\
		\hline
		$]6;\infty[$  & $p_7$ & $p_8$ & $p_9$ \\
		\hline
	\end{tabular}
	\caption{A különböző tartományokban alkalmazott súlyok}
	\label{felosztas}
\end{table}
\noindent
Ha $\sigma_{cs} = 0$ és $n_{cs} = 1$, akkor a súlyfaktor $s = p_{10}$, ha $\sigma_{cs} = 0$ és $n_{cs} > 1$, akkor a súly $s = p_{11}$, $\sigma$ pedig $p_{12}$-vel egyenlő. A $\sigma_{cs} = 0$ eset külön vételére azért van szükség, mert nagyon sok csoportban csak egy embernek ismert az életkora, emiatt annak a szórása $0$. Az $n_{cs}$ ebben az esetben nem a tényleges csoport méret, hanem azoknak az embereknek a száma, akiknek ismert az életkora a csoportban, a vizsgált személyt nem számolva.
A fenti a paraméterek egy $\mathbf{p} = (p_1, p_2, ..., p_{12})$ vektorba rendezhetőek.

% TODO: miért így osztottam fel
A paraméterek fentebb leírt módon való kiválasztása, azért is jó, mert a csoportméretet és a csoportszórást nem csak egymástól függetlenül tudjuk súlyozni, azaz $s(n_i, \sigma_i) \neq s'(n_i)\cdot s''(\sigma_i)$. Ezáltal például megkülönböztethetünk egy osztályt egy munkatársakból álló csoporttól. A csoportméret a kettőnél ugyanúgy nagy lehet, viszont az előbbinek a csoportszórása kisebb, ezért azt gondoljuk, hogy azt nagyobb súllyal kell figyelembe venni.
\begin{figure}[H]
	\centering
	\begin{subfigure}{0.49\textwidth}
		\includegraphics[scale=0.5]{img/group_sizes.eps}
		\caption{Csoportok száma a csoportméret ($n_{cs}$) függvényében a különböző ismerősszámú hálózatok esetében}
	\end{subfigure} \hfill
	\begin{subfigure}{0.49\textwidth}
		\includegraphics[scale=0.5]{img/group_devs.eps}
		\caption{Csoportok száma a csoportszórás ($\sigma_{cs}$) függvényében a különböző ismerősszámú hálózatok esetében}
	\end{subfigure}
	\caption{Csoportok méret- és szóráseloszlásai}
	\label{meret_szoras_eloszlas}
\end{figure}
Abban, hogy a paraméterek határait hol húzzuk meg \aref{meret_szoras_eloszlas}. ábra segített. Megpróbáltunk jellegzetes tartományokat kiválasztani, ebből lett \aref{felosztas}. táblázat szerinti felosztás. Emellett nem szerettünk volna sok paramétert se, mert az optimalizálásnál az nem előnyös. 
% TODO: negatív paraméterek?

\subsection{Sikerességi ráták}
Ha a becslés pontosságát, sikerességét számokban akarjuk kifejezni, akkor a valódi és a becsült életkor közötti különbséget használhatjuk fel. A becslést elvégezzük az összes személyre, akinek tudjuk az életkorát, majd megnézzük, hogy a becslések hány százalékában volt a becsült és a valós korok közti eltérés $\pm 1, 2,3,4,5$ év. Optimalizálatlan esetben a sikerességek \aref{telefon_sikeresseg_nemopt}. és \aref{iwiw_sikeresseg_nemopt}. táblázatban foglaltak szerint alakulnak.
\begin{table}[H]
	\centering
	\setstretch{1.5}
	\begin{tabular}{|c|c|}
		\hline
		Hiba (év) & Ismerősszám: 60-69 \\
		\hline
		$\pm 1$ & 22,7 \% \\
		\hline
		$\pm 2$ & 35,4 \% \\
		\hline
		$\pm 3$ & 45,0 \% \\
		\hline
		$\pm 4$ & 52,2 \% \\
		\hline
		$\pm 5$ & 57,8 \% \\
		\hline
	\end{tabular}
	\caption{Sikerességek a telefonos adatok esetében}
	\label{telefon_sikeresseg_nemopt}
\end{table}
\begin{table}[H]
	\centering
	\setstretch{1.5}
	\begin{tabular}{|c|c|c|c|c|}
		\hline
		\multirow{2}{*}{Hiba (év)} & \multicolumn{4}{|c|}{Ismerősszám} \\ \cline{2-5} & 50 & 100 & 200 & 300 \\
		\hline
		$\pm 1$ & 39,4 \% & 53,2 \% & 63,8 \% & 67,5 \% \\
		\hline
		$\pm 2$ & 48,8 \% & 63,5 \% & 76,4 \% & 80,7 \% \\
		\hline
		$\pm 3$ & 54,5 \% & 68,7 \% & 80,7 \% & 84,8 \% \\
		\hline
		$\pm 4$ & 57,7 \% & 71,7 \% & 83,2 \% & 87,0 \% \\
		\hline
		$\pm 5$ & 60,6 \% & 74,0 \% & 84,9 \% & 88,5 \% \\
		\hline
	\end{tabular}
	\caption{Sikerességek az iWiW-es adatok esetében}
	\label{iwiw_sikeresseg_nemopt}
\end{table}

\subsection{Kiértékelési idő}


\section{Optimalizációs módszerek}
Egyszerű esetben az optimalizáció egy függvény minimum- vagy maximumhelyének megkeresését jelenti egy bizonyos tartományon \cite{Holder}. \iffalse TODO: példa a prog opt.-ról? \fi Ez azért lehet nehéz feladat, mert a függvényt kiértékelni egy pontban sokáig tarthat. Nagyon sok idő lenne az értelmezési tartomány egészére kiértékelni a függvényt.

Ebben a fejezetben különböző optimalizációs módszerek alapját próbáltam ki azzal a céllal, hogy az $E(\mathbf{p})$ függvényt maximalizáljam, tehát az életkorbecslés ($\pm 2$ éven belüli) sikerességi hányadosát növeljem.
\subsection{Gradiens módszer}
% https://en.wikipedia.org/wiki/Gradient_descent
\subsection{Szimulált hűtés}
\subsection{Genetikus algoritmus}
\subsection{Futási idők}
\section{Eredmények}
\section{Összegzés}

%\renewbibmacro{in:}{}
%\printbibliography[title={Irodalomjegyzék}]
\bibliography{bibliography}
\end{document}